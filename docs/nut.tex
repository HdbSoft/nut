\documentclass{article}
\usepackage{listings}

\title{The nut kernel}
\author{\textit{OS development simplified}}
\date{}

\begin{document}
  \maketitle
  \newpage
  \begin{center}
    This is an open source document, you can find it in nut's source code (github.com/HdbSoft/nut)
  \end{center}
  \newpage

  \begin{enumerate}
  	\Large \item Introduction
    \large \subitem What is nut?
    \large \subitem Which languages are supported?
    \large \subitem Is nut free?
  	\\
  	\Large \item Building nut
  	\large \subitem Cloning nut
    \large \subitem Setting up dependencies
    \large \subitem Using the Makefile
  	\large \subitem Building for x86 target
  	\large \subitem ARM \& raspberry Pi target
  	\large \subitem RISC-V target
  	\large \subitem SPARC target
  	\large \subitem MIPS target
  	\large \subitem MOS 6502 target
  	\large \subitem GBA target
    \\
  	\Large \item The nut DLL
  	\large \subitem Abort module
    \large \subitem FAT32 module
    \large \subitem Network module
    \large \subitem nut for Gameboy Advance
    \\
  	\Large \item Nut include files
    \large \subitem Locating include files
    \large \subitem ARM utils
    \normalsize \subitem \hspace{1cm} syscalls.h
    \normalsize \subitem \hspace{1cm} \_file.h
    \large \subitem Gameboy Advance utils
    \normalsize \subitem \hspace{1cm} pixel.h
    \normalsize \subitem \hspace{1cm} scene.h
    \large \subitem alloc.h module
    \large \subitem basics.h module
    \large \subitem colors.h module
    \large \subitem error.h module
    \large \subitem io.h module
    \large \subitem keys.h module
    \large \subitem py.h module
    \large \subitem rpiGPIO.h module
    \large \subitem string.h module
    \large \subitem types.h module
    \large \subitem utils.h module
    \large \subitem vga.h module
    \\
    \Large \item Nut extra modules
    \large \subitem Console UI module
    \large \subitem Sound module
    \large \subitem Cursor module
    \\
  	\Large \item NutScript, script files for nut
  	\large \subitem Comments
    \large \subitem Script properties
    \large \subitem I/O (Print, Sound...)
    \large \subitem Error
    \\
    \Large \item Using nut with Rust
    \large \subitem Getting nut.rs
    \large \subitem Types
    \large \subitem I/O functions
    \large \subitem Memory management
    \large \subitem Error handling
    \\
    \Large \item Nut utils
    \large \subitem GRUB file
    \large \subitem Linker script
  \end{enumerate}
  \newpage

  \section{Introduction}
  Welcome to the nut kernel documentation, in this book, you will learn all the nut kernel features, from building it from x86 to RISC-V, ARM and other platforms, to create scripts and use nut with Rust lang. By now, nut is in an unstable version, with bugs and errors; if you want to use nut in your OS, consider these warnings:
  \begin{itemize}
    \item There could be some errors in your OS.
    \item Your OS will be unstable.
    \item You could have some security failures.
  \end{itemize}
  No more things to say so let's get started!
  \\
  \subsection{What is nut?}
  Nut is and open source kernel that lets you build fast and efficient operating systems for multiple platforms, by now, these are the supported platforms:
  \begin{itemize}
    \item x86
    \item ARM
    \item ARM32
    \item RISC-V
    \item 6502
    \item GBA
    \item raspberry Pi A, B, Zero
    \item raspberry Pi 2
    \item raspberry Pi 3, 4
    \item SPARC
    \item MIPS
  \end{itemize}
  If you want to use nut in another platform, you can adapt it by yourself, for example, if you want to use nut in PowerPC, you can build the boot file and compile the kernel by yourself (our Makefile offers help with every architecture).
  \\
  \subsection{Which languages are supported?}
  By now, nut supports 3 different languages: C, C++ and Rust. If you want to use nut with Rust, our API does not support GBA builds, but you can build your own GBA Rust API.
  \\
  \subsection{Is nut free?}
  Yes, nut is totally free free and open source, no private binaries here, just pure source code made in C/C++, assembly language, Rust, Makefile, YACC and LEX.
  \newpage

  \section{Building nut}
  In this section, you are going to learn how to build nut from source code, using GCC and GNU make. You are also going to learn how to cross compile nut for every supported architecture. You will need GNU/LINUX, macOS or WSL as operating system.
  \\
  \subsection{Cloning nut}
  For this part, you will need an internet connection and it's recommended to have git installed in your device. When you have these requirements, you can finally clone nut using git:
  \begin{lstlisting}[language=bash]
  git clone https://github.com/HdbSoft/nut.git\end{lstlisting}
  If you prefer, you can clone nut with the GitHub client:
  \begin{lstlisting}[language=bash]
  gh repo clone HdbSoft/nut\end{lstlisting}
  Finally, move into the source directory:
  \begin{lstlisting}[language=bash]
  cd nut/\end{lstlisting}
  And you will have nut cloned.
  \\
  \subsection{Setting up dependencies}
  In this part, you must have cloned nut, if you don't, jump to the previous section. Now, just run your OS's dependencies setup (if you are in macOS, make sure you have Homebrew installed):
  \begin{lstlisting}[language=bash]
  bash setup/setup.deb.sh #debian based OSs
  bash setup/setup.dnf.sh #RHEL based OSs
  bash setup/setup.arch.sh #Arch LINUX based OSs
  bash setup/setup.brew.sh #macOS\end{lstlisting}
  If you are in Windows, there is also a setup for you, but you will need to have chocolatey installed and PowerShell scripts enabled. Then, run the setup file:
  \begin{lstlisting}[language=bash]
  powershell setup/setup.win.ps1\end{lstlisting}
  The setup will install the following packages:
  \begin{itemize}
  	\item clang
  	\item bison
  	\item flex
  	\item make
  \end{itemize}
  Now you are ready to build nut.
  \\
  \subsection{Using the Makefile}
  In this part I'm going to try to explain a bit how nut's Makefile works. First, the Makefile declares the following variables:
  \begin{itemize}
  	\item ARCH to x86
  	\item CPUBITS to 64
  	\item CC to clang
  	\item OUTPUT to bin
  	\item NUTSRC to nut.c
  	\item NUTOBJ to nut.so
  	\item BOOTOBJ to boot.o
  	\item ASSEMBLER to as --32
  	\item INCLUDE-DIR to /usr/include
  	\item C2ASM to clang -S
  	\item EXISTS-NUT-INCLUDE to yes
  	\item LATEX to pdflatex
  	\item TO to /dev/null
  \end{itemize}
  Then the Makefile compares if exists the nut directory in /usr/include/ directory and your selected architecture and selects the boot file, the arch file and the C compiler flags. Finally, the recipes do the easy work by compiling nut with you configured variables.
\end{document}